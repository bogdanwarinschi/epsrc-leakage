% =============================================================================

\section{Abstract}
\label{sec:abs}

This travel grant will enable Prof. Warinschi to undertake collaborative work together with Prof. Alexandra Boldyreva (Georgia Tech) and Prof. Geoffrey Smith (Florida International University) in the area of encrypted databases.
Profs. Boldyreva and Smith have recently been awarded a collaborative grant by the US National Science Foudation 
(project EAGER -- NSF award \#1749069) to work on tackling the issue of information leakage in such databases through a combination of cryptographic and information theoretic techniques. 
These ideas arose from a joint meeting between Profs. Boldyreva, Smith, and Warinschi. 


\iffalse
Specifically, the goal is to develop a principled approach to quantifying and understanding the inherent information leakage in such systems.  
The state of the art is, essentially, that 
\fi





% =============================================================================

\section{Previous track record}
\label{sec:prev}

% -----------------------------------------------------------------------------

\iffalse
\subsection{Environment}
\label{sec:prev_env}

The University of Bristol (UoB) Computer Science Department (from here 
on referred to as ``the Department'') is internationally competitive
within the discipline, with $87$\% of research output ranked as $4*$ 
or $3*$ by the $2014$ REF process (which ranked it $4$-th in the UK
based on research intensity).  It hosts a rich, diverse research 
culture, and maintains connections with numerous other disciplines; an
example is the Mathematics Department, via the Heilbronn Institute, a 
partnership between the UoB and GCHQ.  The Department enjoys a close 
relationship with local and global industry, and maintains a portfolio
of ongoing research projects in collaboration with numerous companies.  

Within the Department, the Cryptography and Information Security Group
(from here on referred to as ``the Group'')\footnote{
\url{http://www.cs.bris.ac.uk/Research/CryptographySecurity}
} fosters a internationally leading, interdisciplinary programme of
research that spans theoretical and practical aspects of cryptography 
(and information security more generally); this modus operandi has led 
to numerous advances that would not have been possible by focusing on 
a narrower remit, and is evidenced by output that span {\em all} venues 
of the International Association of Cryptologic Research (IACR), namely 
ASIACRYPT, EUROCRYPT, and CRYPTO, 
{\em plus} the sub-conferences 
CHES, FSE, PKC, and TCC.
At the time of writing, the Group consists of 
 $8$ members of academic staff (plus $2$ visiting fellows, namely Prof. Andy Clark and Prof. Phillip Bond),
 $6$ PDRAs,
and 
$18$ PhD students,
who represent the nucleus of UoB accreditation by EPSRC and GCHQ/NCSC 
under the Academic Centre of Excellence in Cyber Security Research 
(ACE-CSR)\footnote{
\url{http://www.ncsc.gov.uk/articles/academic-centres-excellence-cyber-security-research}
} programme.  
%The affiliated academic staff, presented in no particular order and in 
%addition to the \PI, include:
%
%\begin{compactitem}
%
% \item Prof. Nigel Smart,
%%      (\url{http://scholar.google.com/citations?user=Qvm3k64AAAAJ})
%       was previously supported by 
%       a Royal Society Wolfson Research Merit Award,  
%       and is currently by
%       an EPSRC Advanced Fellowship and an ERC Advanced Grant.
%       His work spans theoretical and practical aspects of public-key 
%       cryptography (especially elliptic curve based techniques), and
%       emerging fields of Multi-Party Computation (MPC) and Fully 
%       Homomorphic Encryption (FHE).
% 
% \item Prof. Elisabeth Oswald 
%%      (\url{http://scholar.google.com/citations?user=eC_kzeUAAAAJ})
%       was previously supported by 
%       an EPSRC Leadership Fellowship,
%       and is currently by
%       an ERC Consolidator Grant.
%       Her work focuses on physical security of computing devices, with
%       a recent emphasis on formalisation of attack and countermeasure
%       techniques and their applicability in evaluation procedures.
% 
% \item Dr. Bogdan Warinschi,
%%      (\url{http://scholar.google.com/citations?user=kS0oatgAAAAJ}), 
%       who leads the group,
%       works at the interface between provable security and symbolic or
%       formal methods; he uses these techniques to address challenges 
%       in protocol analysis (e.g., through composition and abstraction 
%       mechanism) and application in concrete systems (e.g., electronic 
%       voting).
%
%%\item Dr. Daniel Page
%%      (\url{http://scholar.google.com/citations?user=z55uDZcAAAAJ})
%%      works on two broad fields in applied cryptography, namely the
%%      implementation of cryptographic primitives and arithmetic (in 
%%      hardware and/or software), and physical security (relating to 
%%      side-channel and fault attacks, for instance) on embedded and
%%      networked computing devices.
% 
% \item Dr. Martijn Stam 
%%      (\url{http://scholar.google.com/citations?user=2uRr5-YAAAAJ})
%       has a strong background in theoretical and practical aspects of 
%       cryptography, including provable security; his current focus is
%       high-level design and analysis of symmetric primitives (e.g.,
%       cryptographic hash functions, and AEAD-like constructions).
%
% \item Dr. Theo Tryfonas 
%%      (\url{http://scholar.google.com/citations?user=lHYkddUAAAAJ})
%       works in security engineering, and especially digital forensics.
%       A major current focus is the security challenges stemming from
%       ``smart cities'' (including smart-grid, SCADA, and IoT more
%       generally), and networks of autonomous vehicles.
% 
% \item Dr. Georgios Oikonomou
%%      (\url{http://scholar.google.com/citations?user=oLr89jwAAAAJ})
%       has a focus on sensor networks (including IoT generally, e.g.,
%       as part of the SPHERE project) and digital forensics; he is a 
%       core developer of the Contiki OS.
% 
% \item Dr. David Bernhard
%       is currently a Senior Teaching Associate within the Department,
%       although he maintains a research profile relating to the design 
%       and security of electronic voting (and associated cryptographic 
%       technologies that support or relate to it).
%
%\end{compactitem}

The Group is guided by a dedicated Industrial Advisory Board (IAB), and
supported by a diverse range of funding streams (e.g., industry, EPSRC, 
EU, DARPA).  Various mature, well supported research platforms support 
associated activity: for example
a) the Group maintains 
   lab. space used for 
   hardware development  (e.g., FPGA development, PCB prototyping)
   and
   side-channel analysis (e.g., power- and EM-acquisition and post-processing),
b) UoB       maintains 
   a centralised High-Performance Computing (HPC) facility organised 
   under the Advanced Computing Research Centre (ACRC); $\pounds 12$ 
   million invested since $2006$ has provided dedicated, managed
   facilities for both computation and storage.

% -----------------------------------------------------------------------------

\subsection{Personnel}
\label{sec:prev_people}
\fi

\paragraph{Prof. Bogdan Warinschi} joined the University of 
Bristol as a Lecturer in the Computer Science Department in February 2007. 
He has a Ph.D. in Computer Science from the University of California
at San Diego, and was previously affiliated as a postdoctoral researcher with
University of California at Santa Cruz, Stanford University, and the
French National Research Institute in Informatics (INRIA).
He conducts research into security and cryptography, with a particular emphasis on
proof methods for the security of protocols.


Dr. Warinschi served on more than thirty program committees of top conferences, most recently on Security and Privacy '16, Eurocrypt'15 and Eurocrypt'14.  He is internationally known for his roly in establishing the research area generically known as {\em computational   soundness}~\cite{cortier05computationally,micciancio04soundness}.
%The basic idea is to combine two fundamentally different paradigms for proving security, one based on symbolic (formal methods-like) techniques and one based on complexity and computation theory. 
%The main benefit of the approach is that it enables simpler analysis which uses higher level abstractions yet which offers computational security guarantees. 

This proposal relies on the expertise of Prof. Warinschi in the design of security models and his interest in analysing systems used in practice. In particular, he has maintained a constant interest in designing cryptographic security models, the first step towards the rigorous analysis of any cryptographic system. 
His research addresses the security of group signatures \cite{BMW03}, proxy signatures \cite{proxies}, defence mechanisms for DOS resistance (cryptographic puzzles) \cite{puzzles},  key-exchange \cite{ke}, voting schemes \cite{helios}, and cryptographic APIs \cite{KSW11}.
This work is complemented by security analysis on protocols and primitives that are deployed.   These include Public Key Infrastructures (PKI) \cite{boldyreva07acloser}, the Trusted Platform Module (TPM) \cite{pcas}, the Helios Internet voting schemes \cite{helios}, and the ubiquitous Transport Layer Security protocol (TLS) \cite{tls}. 

His recent research interests are in understanding and enabling the use of cryptography in practical scenarios.  This travel grant will support his participation in a project that falls within this broad direction.

%To date, eight students have completed their PhD under the (co-)supervision of Prof.~Warinschi; he is currently (co-)supervising two students.

\paragraph{Prof. Alexandra Boldyreva}
Prof. Boldyreva is an Associate Professor at Georgia Tech Institute for Information Security \& Privacy.  Among her main areas of research she is a world leading expert in the security of encryption schemes with additional functionality properties.
 In particular she has helped establish foundations for deterministic, order-preserving and searchable encryption ~\cite{bellare2007deterministic,boldyreva2009order,boldyreva2011order,boldyreva2008notions,boldyreva2014efficient}.


\paragraph{Prof. Geoffrey Smith}
Prof. Smith is a Professor at the School of Computing and Information Sciences, Florida International University.  He has made seminal contributions to the area of quantitative information flow \cite{volpano1996sound,smith2009foundations,volpano1999probabilistic}. 
Relevant for this proposal, Prof. Smith is the main author of a series of papers which lays foundations to \emph{generalized gain functions}, a generalization of information entropy \cite{m2012measuring,mciver2014abstract,alvim2014additive,alvim2016axioms}. 
One of his papers in this area has received the ``Best Cybersecurity Research Paper'' award offered by the U.S. National Security Agency~\cite{alvim2014additive} \footnote{\url https://www.nsa.gov/news-features/press-room/press-releases/2015/annual-cyber-research-paper-comp-winner.shtml}.



% =============================================================================
