% =============================================================================

\section{Background}
\label{sec:prop_bg}

Cloud storage is currently experiencing explosive growth. Users regularly rely on the cloud to store businesscritical
information, and the average organization uploads 13.9 TB of data to the cloud each month [3].
With all of this data stored remotely, security of data is becoming a core requirement for any organization looking to leverage the public cloud.  

In this setting, even if one is willing to trust the cloud, it is crucial to also guarantee security against intrusion attacks or dishonest employee actions. 
Ensuring this level of security without compromising functionality and efficiency is one of the major barriers in the 
face of wider cloud storage adoption, especially when compliance with legal requirements such as (the US regulation comprised by) PCI DSS and HIPAA, as well as privacy requirements under laws such as the EU Data Protection Regulation. 

It is not hard to observe that simply using off-the-shelf encryption schemes does not “work” for cloud
storage application. This is because standard encryption is so strong as to hide \emph{all} poartial information about the plaintexts, so whoever stores the ciphertexts cannot implement \emph{any} functionality  which depends on the underlying plaintexts. 

\paragraph{Computing on encrypted data}
To enable cloud-based applications which go beyond simply storing data, it is necessary to enable the cloud to perform computations on encrypted data in a way that does not completely sacrifice privacy or security for utility.
This research direction had attracted a lot of interest and led to theoretical solutions like Oblivious RAM (ORAM) [33, 23, 35] or fully homomorphic encryption (FHE) [22] which offer the desired functionality and leak no information about the underlying data. 
Unfortunately, at present it seems that complete privacy for the underlying data inherently leads to impractical solutions. 

A more pragmatic approach proposes to design encryption schemes which leak \emph{some} information, sufficient to enable running algorithms on top of encrypted data, while revealing as little as possible about what exactly is encrypted. 
The practical implications for secure solutions for this problem are staggering and have attracted a significant amount of research.
Consider for example  the rather restricted functionality offered by symmetric searchable encryption.
There is by now a plethora of academic research on the topic \cite{}. 
The solutions proposed differ in the level of functionality, security and efficiency they provide: 
these properties are almost always at odds with each other. 
Most of the schemes address the exact-match keyword search, but there are also solutions for enabling
sorting the encrypted data, performing conjunctive, Boolean, range, similarity, substring queries, etc. Some
protocols require a user or a proxy to create an encrypted index of the data and send it to the server along
with encrypted data. Some are in the public- and some are in the symmetric-key settings. Some solutions
are transparent in that the server can handle ciphertexts in the exact same way it searched unencrypted data,
so no server-side modifications are required. The existing solutions also vary greatly security-wise. This
is not surprising given that one often must compromise some security to achieve better efficiency or
functionality, and there is no trade-off which is right for all circumstances. 

\paragraph{The problem}
All of the above searchable encryption schemes and, more generally, schemes where information is leaked in exchange of improved functionality share a major shortcomming: the security implications of the information that is disclosed are only rarely clear. 
Very roughly, security definitions for such schemes formalize the information leaked as  a function which maps the secret information that should be protected to a bitstring.  A scheme is deamed secure if any attack against the scheme can be simulated by an adversary who only has access to the leaked information.

The leakage function is therefore a parameter of the definition but there are no restrictions imposed on it. 
Specifying the leakage function associated to a scheme is left to the scheme designer, and determining what are acceptable functions is left to the users of the scheme.
For some functions, e.g. a function that returns the first field of a database record, it may be clear and relatively simple to determine if the risc is acceptable.  However, only informal arguments where attacks that exhibit obviously insecure behaviour have appeared in the literature. 
More damaging, several heuristic attacks show that information leakage for these schemes can be used to reveal highly sensitive information which is not obviously leaked~\cite{cash2015leakage}. 

This uncomfortable state of affairs shows that what is missing is a principled approach to the problem of understanding and quantifying information leakage in this type of schemes. 

% -----------------------------------------------------------------------------
\newpage

\section{Research project}
\label{sec:prop_hyp}

The \PI together with Prof. Boldyreva and Prof. Smith have identified a plausible approach to solving this problem through a mix of cryptographic and information theoretic techniques. 
The idea is rooted in work by the PI in the context of electronic voting~\cite{bernhard2012measuring}. 
There, it is observed that no matter how well individual votes are proected, the result of the voting process necessarily leaks information about the votes.
The simplest way to see this is for (the unlikely but possible) situation where some candidate is voted unanimously; if this information is revealed then it is clear how everyone voted.  More generally, information regarding how votes are distributed among candidates imply that \emph{some} information is leaked about individual votes.  Privacy definitions for voting are usually not concerned with this type of leak, much like those for searchable encryption abstract away leakage profiles. 

% For voting systems this problem has been addressed by the \PI~\cite{bernhard2012measuring}. 
The main result of ~\cite{bernhard2012measuring} shows how to use various notions of \emph{computational entropy} to measure the information that is revealed to the adversary: the work provides a mapping between a quantitative measure of lost information as captured by several different notions of entropy (Hartley, min, average, conditional) and the success probability of several distinct type of attackers. 

\paragraph{Research hypothesis}
The research hypothesis behind this project is that, for searchable encryption,  a similar approach can provide quantitative measures to measure information leakage; these notions should be directly related to informative adversarial models which clarify the adversarial goals. 

\paragraph{Collaboration}
The project will use generalized gain functions, or \emph{g-leakage} in short, to develop a quantitative measure for the vulnerability induced by the (allowed) leakage in encrypted databases. 
 This concept was introduced and refined in a series of papers co-authored by Prof. Smith ~\cite{DBLP:conf/csfw/AlvimCPS12,DBLP:conf/csfw/AlvimCMMPS14,DBLP:conf/csfw/AlvimCMMPS16,DBLP:conf/csfw/SmithS17} as 
a generalization of information theoretic entropy. 
What makes it suitable for our purpose is that it comes with an operational interpretation which links (generalizations of) entropy with attacker models. 

% the US National Security Agency awarded in 2015 the Best Cybersecurity Research Paper to one of the papers in the series \footnote{\url https://www.nsa.gov/news-features/press-room/press-releases/2015/annual-cyber-research-paper-comp-winner.shtml}. 

The \PI had suggested to Prof. Boldyreva to use g-leakage approach to measure information leakage in encryption schemes, following which a 3-day meeting with the \PI and Prof. Smith in April 2015.  In the meeting several interesting research directions were identified; one of them was to use the setting of searchable encryption as a test case. 

This line of research has recently been granted support by the US National Science Foundation who has recently awarded a grant for collaborative work (\textit{EAGER: Collaborative: Quantifying Information Leakage in Searchable Encryption} -- Award \#1749069) for developing a principled understanding of information leakage in searchable encryption. 
This grant will run between January 2018 and June 2019.  

Most of the interaction between the \PI and the EAGER partners will be virtual (e.g. joint Skype meetings).  
However, focussed research meetings are important for rapid and decisive progress.  
This grant will support visits by Prof. Warinschi to project partners as well as participation to joint research meetings. 

\paragraph{Timeliness and impact}
Recent high-profile attacks on encrypted databases~\cite{naveed2015inference,cash2015leakage} have exposed 
%These attacks target existing proposals with mathematical security proofs; some of the proposals are actually implemented and deployed by start-ups based in the US. 
a large gap between the advertised security guarantees for encrypted databases and those required for secure deployment in cloud environments. 
Most attacks 

These attacks show the necessity for a principled understanding of the issues that surround information leakage in this settings as an important first step to filling this gap. 
The US funded \emph{EAGER} which brings together world leading experts in encrypted databases and 

project therefore offers a unique opportunity to develop 

is project proposes will take an important first step in this direction, with the aim to enable cloud technologies for storing and processing data with clear risk profiles. 



\paragraph{Outlook}
Encryption schemes that trade some security for increased functionality come in many different shapes and forms. 
An attempt to unify their syntax and security is the concept of property preserving encryption. 
It comprises, (among its many different instantiations) several high-profile types of encryption schemes with different flavours of functionality: deterministic encryption reveals duplicates, order preserving encryption allows for efficient search in linearly ordered sets, fuzzy search allows for searching for approximate forms of keywords.   

While the current project will develop a framework for coping with information leakage in searchable encryption, we envision that this framework will extend 



% -----------------------------------------------------------------------------

\section{Past and ongoing collaborations}
\label{sec:prop_method}
The the Prof. Warinschi and Prof. Boldyreva have a history of sporadic but highly successful collaboration that spans more than a decade. 
Some of their joint research papers~\cite{boldyreva2007closer,boldyreva2009foundations,boldyreva2012secure,lipton2016provably} have helped established new research directions, as evidenced by their high citation counts. 

Currently, they are collaborating to 

The importance of information leakage for encryption schemes have arouse in the context of recent joint work in the area of encryption for machine learning; one co-authored paper is currently under submission. 






% -----------------------------------------------------------------------------


% =============================================================================



